% !TEX TS-program = pdflatex
% !TEX encoding = UTF-8 Unicode

% This is a simple template for a LaTeX document using the "article" class.
% See "book", "report", "letter" for other types of document.

\documentclass[11pt]{article} % use larger type; default would be 10pt

\usepackage[utf8]{inputenc} % set input encoding (not needed with XeLaTeX)

%%% Examples of Article customizations
% These packages are optional, depending whether you want the features they provide.
% See the LaTeX Companion or other references for full information.

%%% PAGE DIMENSIONS
\usepackage{geometry} % to change the page dimensions
\geometry{a4paper} % or letterpaper (US) or a5paper or....
% \geometry{margin=2in} % for example, change the margins to 2 inches all round
% \geometry{landscape} % set up the page for landscape
%   read geometry.pdf for detailed page layout information

\usepackage{graphicx} % support the \includegraphics command and options

% \usepackage[parfill]{parskip} % Activate to begin paragraphs with an empty line rather than an indent

%%% PACKAGES
\usepackage{booktabs} % for much better looking tables
\usepackage{array} % for better arrays (eg matrices) in maths
\usepackage{paralist} % very flexible & customisable lists (eg. enumerate/itemize, etc.)
\usepackage{verbatim} % adds environment for commenting out blocks of text & for better verbatim
\usepackage{subfig} % make it possible to include more than one captioned figure/table in a single float
\usepackage{graphicx} % can include graphics
\usepackage{float}
\usepackage{listings}
\usepackage{url}
\usepackage[spanish]{babel}
% These packages are all incorporated in the memoir class to one degree or another...

%%% HEADERS & FOOTERS
\usepackage{fancyhdr} % This should be set AFTER setting up the page geometry
\pagestyle{fancy} % options: empty , plain , fancy
\renewcommand{\headrulewidth}{0pt} % customise the layout...
\lhead{}\chead{}\rhead{}
\lfoot{}\cfoot{\thepage}\rfoot{}

%%% SECTION TITLE APPEARANCE
\usepackage{sectsty}
\allsectionsfont{\sffamily\mdseries\upshape} % (See the fntguide.pdf for font help)
% (This matches ConTeXt defaults)

%%% ToC (table of contents) APPEARANCE
\usepackage[nottoc,notlof,notlot]{tocbibind} % Put the bibliography in the ToC
\usepackage[titles,subfigure]{tocloft} % Alter the style of the Table of Contents
\renewcommand{\cftsecfont}{\rmfamily\mdseries\upshape}
\renewcommand{\cftsecpagefont}{\rmfamily\mdseries\upshape} % No bold!

%%% END Article customizations

%%% The "real" document content comes below...

\title{Lenguaje de programacion PHP}
\author{Lozano E. ,Lasso H. \& Alvarado J.}
%\date{} % Activate to display a given date or no date (if empty),
         % otherwise the current date is printed 

\begin{document}
\maketitle

\section{Introducción}
PHP es un lenguaje de programación definido para el desarrollo web de contenido dinámico y que se caracteriza por ser de código abierto. Sus siglas son un acrónimo recursivo de: “PHP: Hypertext Preprocessor” y actualmente ya en su quinta versión (PHP 5).
\\¿Qué hace a PHP importante? este lenguaje se ejecuta justo antes de enviar una página desde el servidor web a la ventana del usuario como vemos en la Figura \ref{fig:funcionamiento}, y su respuesta es una página en código HTML, he allí su compatibilidad con todos los navegadores, sistemas operativos y plataformas.  Es fácil de utilizar y con grandes ventajas como la gratuidad, independencia de plataforma, rapidez y seguridad. 

\begin{figure}[H]
  \centering
    \includegraphics[width=0.4\textwidth]{Imagenes/diagrama-php}
  \caption{Esquema del funcionamiento de las páginas PHP}
  \label{fig:funcionamiento}
\end{figure}
 Y aunque este lenguaje tiene muchas ventajas, aun no se encuentra disponible un IDE con el cual mejorar su usabilidad; aunque igual trabaja muy bien y ha tomado así el papel principal del scripting (un lenguaje scripting es un tipo de lenguaje de programación que es generalmente interpretado) \cite{[1]}.

\section{Características}

\begin{itemize}
\item Puede ser utilizado  en diversas plataformas, es decir que es multiplataforma.
\item Ofrece diferentes estrategias para manejar excepciones.
\item Posee una amplia biblioteca nativa de funciones.
\item Permite técnicas de programación orientada a objetos.
\item Orientado a desarrollo de aplicaciones web dinámicas.
\item Todas las funciones del sistema están explicadas en un  archivo de ayuda.
\item Es un software libre, es decir de fácil acceso para todos.
\item Puede conectarse con la mayoría de los motores de bases de datos que se utilizan en la actualidad.
\item Es seguro y confiable debido a que su código fuente es invisible tanto al navegador como al cliente.
\end {itemize}

\section{Historia}

Su creador fue  Rasmus  Lerdorf  quien nació en  Groenlandia  y cuya  profesión fue Ingeniero en Diseño de Sistemas Informáticos.En el año de 1994 Rasmus creó la primera versión de PHP, lo que quería era saber cuántas personas estaban leyendo su curriculum vitae en 
su página web , para poder realizar esto tuvo que crear  un   “Common Gateway Interface” en  Perl  que le revelaba  los resultados estadísticos en su propio sitio web.Rasmus el 8 de Junio de 1995 le puso como nombre a ese script PHP  “acrónimo de Personal Home Page”.

Dada a su gran aceptación Ramus decidió crear  un sistema para procesar formularios al que le puso el nombre de From Interpreter, y la unión de estas dos herramientas  se da  a cabo  la primera versión compacta del lenguaje :PHP/FI, cuya característica principal fue soporte interno para bases de datos, cookies y funciones definidas por el usuario.Una de las características de esta versión de PHP es proveer a los usuarios  una interfaz madura para múltiples bases de datos, protocolos, y APIs.

En 1997 debido a que PHP/FI era ineficiente y carente de las características que se requerían para impulsar una aplicación de comercio electrónico que estaban desarrollando para un proyecto de universidad.Andi Gutmans y Zeev Suraski discutieron varios aspectos de la implementación actual de PHP, decidieron mejorar el motor y comenzar a construir sobre las bases de PHP/FI existente, dando lugar a la creación de un nuevo e independiente lenguaje de programación este nuevo lenguaje fue publicado con el nombre de Hypertext  Preprocessor que es una nueva versión de PHP (PHP 3.0), en junio de 1998  PHP 3.0 fue anunciado por el nuevo equipo de Desarrollo de PHP como el descendiente oficial de PHP/FI 2.0.

En el invierno de 1998 Andi Gutmans y Zeev Suraski empezaron a trabajar en una nueva version de PHP, donde desarrollaron un nuevo motor llamado Motor Zend que viene de los nombres de pila Zeev y Andi.Una de la características importantes de esta versión es el soporte para la mayoría de servidores web, buffers de salida, sesiones HTTP y formas más seguras de controlar las entradas de usuarios.

Esta versión llamada PHP 4.0 fue publicada en Mayo del 2000 aproximadamente  dos años después que salió la versión PHP 3.0.
En la siguiente versión de PHP se trato de mejorar los mecanismos de la programación orientada a objetos dando como resultado un lenguaje más potente y que cada vez se hizo más popular, esta versión fue publicada en el año 2004 y se la llamo como PHP 5.0 
 

\section{Tutorial de Instalación}


Antes de empezar con la instalación, primero necesita saber para qué quiere utilizar PHP. Existen tres campos principales donde se puede utilizar PHP:

\begin {itemize}

\item Aplicaciones web y sitios web
\item Scripting en la línea de comandos
\item Aplicaciones de escritorio
\end {itemize}

Para la primera forma se necesita: un servidor web, PHP,  y un navegador web.Dependiendo de la configuración del sistema operativo, quizá ya tenga un servidor web. También puede contratar el servicio de web Hosting en una empresa, de esta forma, no es necesario instalar PHP, simplemente se crean los scripts PHP, se los sube al servidor y se verifica por medio de un navegador web.\\

Para usar PHP bajo la línea de comandos (p.ej. escribir scripts que autogeneran imágenes de forma offline, o procesar ficheros de texto dependiendo de los argumentos que se les pasen),  se necesitará el ejecutable de línea de comandos. En este caso, no se necesita ningún servidor o navegador.\\

Con PHP también se pueden escribir aplicaciones GUI de escritorio usando la extensión PHP-GTK. Este enfoque no tiene nada que ver con escribir páginas web, ya que no se muestra nada de HTML, pero gestiona ventanas y objetos dentro de ellas. PHP-GTK no está incluido en la distribución oficial de PHP.

\subsection{Instalacion Sistemas Unix}

Existen varias maneras de instalar PHP para la plataforma Unix, ya sea con un proceso de compilar y configurar, o a través de varios métodos pre-empaquetados. Usar una version pre-empaquetada puede ayudar en preparar una configuración standard, pero si se requiere tener un conjunto diferente de características (tales como un servidor seguro, o un manejador diferente de base de datos), podría ser necesario construir PHP y/o el servidor web. Si no se está familiarizado con la construcción y el compilado de su propio software, vale la pena revisar para ver si alguien ya ha construido una versión empaquetada de PHP con las características que se necesitan.

\subsection{Instalacion en sistemas Windows}

\begin{enumerate}
\item Instale el servidor web deseado.\\
Para Instalar Apache siga las siguientes instrucciones:
\begin{enumerate}[a)]
\item Descargue Apache desde el sitio Oficial (\url{http://httpd.apache.org/download.cgi})
\item Siga las Instrucciones del Instalador
\begin{center}
\includegraphics[scale=0.7]{Imagenes/Apache01.jpg}
\end{center}
\item Configure la informacion del servidor. Para utilizar Apache como un servidor web local se debe colocar  como el dominio de red ``localhost''; el nombre del dominio puede ser cualquier string. Tipicamente el Email del administrador se coloca como ``root@localhost''
\begin{center}
\includegraphics[scale=0.7]{Imagenes/Apache02.jpg}
\end{center}
\item Si desea personalizar los componentes a instalar seleccione ``Custom'', si no``Typical''
\begin{center}
\includegraphics[scale=0.7]{Imagenes/Apache03.jpg}
\end{center}
\item Al terminar de configurar la Instalacion de clic a Instalar
\begin{center}
\includegraphics[scale=0.7]{Imagenes/Apache04.jpg}
\end{center}
\item Durante el proceso de instalacion de Apache se abriran varias ventanas de DOS, debido a las configuraciones en la linea de comando.
\begin{center}
\includegraphics[scale=0.7]{Imagenes/Apache05.jpg}
\end{center}
\item De click en finalizar para terminar la instalacion
\begin{center}
\includegraphics[scale=0.7]{Imagenes/Apache06.jpg}
\end{center}
\item Para probar que Apache se instalo correctamente vaya a su explorador web y entre a la pagina ``http://localhost'', la cual mostrará lo siguiente
\begin{center}
\includegraphics[scale=0.7]{Imagenes/Apache07.jpg}
\end{center}

\end{enumerate} 

\item Ejecute el instalador MSI(\url{http://windows.php.net/download/#php-5.5}) y siga las instrucciones que le indica el asistente de instalación.
\begin{center}
\includegraphics{Imagenes/install01.jpg}
\end{center}

\item Seleccione el Servidor Web que desea configurar. (Este ya debe estar instalado previamente)
\begin{center}
\includegraphics{Imagenes/install02.jpg}
\end{center}

\item Indique qué funcionalidades y extensiones desea instalar y habilitar. Se recomienda no instalar todas las extensiones de forma predeterminada, ya que muchas de ellas tienen dependencias con bibliotecas externas a PHP.
\begin{center}
\includegraphics{Imagenes/install03.jpg}
\end{center}
\end{enumerate}

Finalmente, el instalador inicializa tanto el fichero php.ini como al propio PHP para funcionar en Windows. También podrá configurar algunos servidores web para que utilicen PHP. Actualmente es capaz de configurar IIS, Apache, Xitami y Sambar; si utiliza otro servidor web, deberá configurarlo a mano.

\section{Hola Mundo y otros Programas Introductorios}
PHP es manejable cuando entendemos programación en HTML, ya que maneja conceptos conocidos como la cabecera (header), el cuerpo (body), etiquetas de apertura y cierre, entre otros. Y así seria un "Hola mundo", en PHP:

\begin{lstlisting}[frame=single]  % Start your code-block

<html>
    <head>
        <title>Ejemplo</title>
    </head>
    <body>
        <?php
            echo "Hola mundoooooo :D";
        ?>
    </body>
</html>
\end{lstlisting}
\begin{center}
Esquema del funcionamiento de las páginas PHP \cite{[2]}
\end{center}
Y así lo veriamos:
\begin{figure}[H]
  \centering
    \includegraphics[width=0.4\textwidth]{Imagenes/HolaMundoPHP-Navegador}
  \caption{Un Hola Mundo con PHP}
  \label{fig:funcionamiento}
\end{figure}

\section{Alcance de la investigación sobre PHP}
El alcance de nuestra investigación fue completar un estudio sobre el lenguaje de Programación PHP, en este artículo se presenta una Introducción que permita entender la funcionalidad básica de PHP, las características del lenguaje que describan su relevancia, la historia de su creación para llegar al punto actual de desarrollo, un tutorial de instalación en maquinas con linux y windows y por ultimo como presentar un 'Hola Mundo' en PHP.\\
En esta investigación también formo parte el estudio del lenguaje en Latex y el manejo de Git para el versionamiento de código en un repositorio. Lo investigado fue hallado en Internet y no se encuentra citado ninguna fuente sobre ello, es decir que a futuro sería bueno en los próximos articulos referenciar las fuentes que nos permitieron entender las herramientas de ayuda que usamos.\\
Fue importante el trabajo grupal ya que los temas estuvieron repartidos entre tres personas (los autores), cada uno se encargo de la investigación respectiva desde su casa, descargando el repositorio, modificandolo y luego volviendolo a subir a Git. 
Hasta el momento la funcionalidad del código es la solicitada. 

\begin{thebibliography}{99}

\bibitem{[1]}  \textit Definición del Lenguaje Scripting [Online]. Available: \url{http://www.alegsa.com.ar/Dic/lenguaje%20scripting.php} Verificado: Octubre 21 - 2013
\bibitem{[2]}  \textsc{ PHP Group} \textit '¿Que es PHP?' [Online].Available: \url{http://php.net/manual/es/intro-whatis.php}  Verificado: Octubre 21 - 2013
\bibitem{[3]}  \textsc{ PHP Group} \textit Historia de PHP [Online]. Available \url{http://php.net/manual/es/history.php.php}  Verificado: Octubre 21 - 2013
\bibitem{[4]}  \textsc{ EcuRed} \textit PHP [Online]. Available \url{http://www.ecured.cu/index.php/PHP}  Verificado: Octubre 21 - 2013
\bibitem{[5]}  \textsc{Installing Apache on Windows} \url{http://www.ross.ws/content/installing-apache-windows}  Verificado: Octubre 23 - 2013
\bibitem{[6]}  \textsc{PHP,Instalación y configuración} \url{http://www.php.net/manual/es/install.php}  Verificado: Octubre 21 - 2013

\end{thebibliography}

\end{document}